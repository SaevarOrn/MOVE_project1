%%%%%%%%%%%%%%%%%%%%%%%%%%%%%%%%%%%%%%%%%
% Programming/Coding Assignment
% LaTeX Template
%
% This template has been downloaded from:
% http://www.latextemplates.com
%
% Original author:
% Ted Pavlic (http://www.tedpavlic.com)
%
% This template uses a Perl script as an example snippet of code, most other
% languages are also usable. Configure them in the "CODE INCLUSION 
% CONFIGURATION" section.
%
%%%%%%%%%%%%%%%%%%%%%%%%%%%%%%%%%%%%%%%%%

%----------------------------------------------------------------------------------------
%	PACKAGES AND OTHER DOCUMENT CONFIGURATIONS
%----------------------------------------------------------------------------------------

\documentclass{article}
%\documentclass[english,a4paper,twoside]{amsart}

\usepackage{fancyhdr} % Required for custom headers
\usepackage{lastpage} % Required to determine the last page for the footer
\usepackage{extramarks} % Required for headers and footers
\usepackage[usenames,dvipsnames]{color} % Required for custom colors
\usepackage{graphicx} % Required to insert images
\usepackage{listings} % Required for insertion of code
\usepackage{courier} % Required for the courier font

\usepackage[utf8]{inputenc}
\usepackage[T1]{fontenc}

\usepackage{amsmath}
\newcommand*{\maxeq}{\stackrel{\text{min}}{=}}
\newcommand*{\mineq}{\stackrel{\text{max}}{=}}

% Margins
\topmargin=-0.45in
\evensidemargin=0in
\oddsidemargin=0in
\textwidth=6.5in
\textheight=9.0in
\headsep=0.25in

\linespread{1.1} % Line spacing

% Set up the header and footer
\pagestyle{fancy}
\lhead{\hmwkAuthorName} % Top left header
\chead{\hmwkClass : \hmwkTitle} % Top center head
\rhead{\firstxmark} % Top right header
\lfoot{\lastxmark} % Bottom left footer
\cfoot{} % Bottom center footer
\rfoot{Page\ \thepage\ of\ \protect\pageref{LastPage}} % Bottom right footer
\renewcommand\headrulewidth{0.4pt} % Size of the header rule
\renewcommand\footrulewidth{0.4pt} % Size of the footer rule

\setlength\parindent{0pt} % Removes all indentation from paragraphs

%----------------------------------------------------------------------------------------
%	CODE INCLUSION CONFIGURATION
%----------------------------------------------------------------------------------------

\definecolor{MyDarkGreen}{rgb}{0.0,0.4,0.0} % This is the color used for comments
\lstloadlanguages{Perl} % Load Perl syntax for listings, for a list of other languages supported see: ftp://ftp.tex.ac.uk/tex-archive/macros/latex/contrib/listings/listings.pdf
\lstset{%language=Perl, % Use Perl in this example
        frame=single, % Single frame around code
        basicstyle=\small\ttfamily, % Use small true type font
        keywordstyle=[1]\color{Blue}\bf, % Perl functions bold and blue
        keywordstyle=[2]\color{Purple}, % Perl function arguments purple
        keywordstyle=[3]\color{Blue}\underbar, % Custom functions underlined and blue
        identifierstyle=, % Nothing special about identifiers                                         
        commentstyle=\usefont{T1}{pcr}{m}{sl}\color{MyDarkGreen}\small, % Comments small dark green courier font
        stringstyle=\color{Purple}, % Strings are purple
        showstringspaces=false, % Don't put marks in string spaces
        tabsize=5, % 5 spaces per tab
        %
        % Put standard Perl functions not included in the default language here
        morekeywords={rand},
        %
        % Put Perl function parameters here
        morekeywords=[2]{on, off, interp},
        %
        % Put user defined functions here
        morekeywords=[3]{test},
       	%
        morecomment=[l][\color{Blue}]{...}, % Line continuation (...) like blue comment
        numbers=left, % Line numbers on left
        firstnumber=1, % Line numbers start with line 1
        numberstyle=\tiny\color{Blue}, % Line numbers are blue and small
        stepnumber=5 % Line numbers go in steps of 5
}

% Creates a new command to include a perl script, the first parameter is the filename of the script (without .pl), the second parameter is the caption
\definecolor{background}{rgb}{0.98, 1, 1}
\newcommand{\shellcmd}[1]{
	\begin{center}
	\texttt{\# cwb-nc> #1}\\
	\end{center}
}
\newcommand{\perlscript}[2]{
\begin{itemize}
	\item[]\lstinputlisting[backgroundcolor=\color{background},stepnumber=1,caption=#2,label=#1]{#1}
\end{itemize}
}

%----------------------------------------------------------------------------------------
%	DOCUMENT STRUCTURE COMMANDS
%	Skip this unless you know what you're doing
%----------------------------------------------------------------------------------------

% Header and footer for when a page split occurs within a problem environment
\newcommand{\enterProblemHeader}[1]{
\nobreak\extramarks{#1}{#1 continued on next page\ldots}\nobreak
\nobreak\extramarks{#1 (continued)}{#1 continued on next page\ldots}\nobreak
}

% Header and footer for when a page split occurs between problem environments
\newcommand{\exitProblemHeader}[1]{
\nobreak\extramarks{#1 (continued)}{#1 continued on next page\ldots}\nobreak
\nobreak\extramarks{#1}{}\nobreak
}

\setcounter{secnumdepth}{0} % Removes default section numbers
\newcounter{homeworkProblemCounter} % Creates a counter to keep track of the number of problems

\newcommand{\homeworkProblemName}{}
\newenvironment{homeworkProblem}[1][Problem \arabic{homeworkProblemCounter}]{ % Makes a new environment called homeworkProblem which takes 1 argument (custom name) but the default is "Problem #"
\stepcounter{homeworkProblemCounter} % Increase counter for number of problems
\renewcommand{\homeworkProblemName}{#1} % Assign \homeworkProblemName the name of the problem
\section{\homeworkProblemName} % Make a section in the document with the custom problem count
\enterProblemHeader{\homeworkProblemName} % Header and footer within the environment
}{
\exitProblemHeader{\homeworkProblemName} % Header and footer after the environment
}

\newcommand{\problemAnswer}[1]{ % Defines the problem answer command with the content as the only argument
\noindent\framebox[\columnwidth][c]{\begin{minipage}{0.98\columnwidth}#1\end{minipage}} % Makes the box around the problem answer and puts the content inside
}

\newcommand{\homeworkSectionName}{}
\newenvironment{homeworkSection}[1]{ % New environment for sections within homework problems, takes 1 argument - the name of the section
\renewcommand{\homeworkSectionName}{#1} % Assign \homeworkSectionName to the name of the section from the environment argument
\subsection{\homeworkSectionName} % Make a subsection with the custom name of the subsection
\enterProblemHeader{\homeworkProblemName\ [\homeworkSectionName]} % Header and footer within the environment
}{
\enterProblemHeader{\homeworkProblemName} % Header and footer after the environment
}

%----------------------------------------------------------------------------------------
%	NAME AND CLASS SECTION
%----------------------------------------------------------------------------------------

\newcommand{\hmwkTitle}{Miniproject\ \#1} % Assignment title
\newcommand{\hmwkClass}{Modeling and Verification} % Course/class
\newcommand{\hmwkClassInstructor}{Anna Ingolfsdóttir} % Teacher/lecturer
\newcommand{\hmwkAuthorName}{Þröstur and Sævar} % Your name

%----------------------------------------------------------------------------------------
%	TITLE PAGE
%----------------------------------------------------------------------------------------

\title{
\vspace{2in}
\textmd{\textbf{\hmwkClass:\ \hmwkTitle}}\\
\vspace{0.1in}\large{\textit{\hmwkClassInstructor}}
\vspace{3in}
}

\author{\textbf{\hmwkAuthorName}}
\date{} % Insert date here if you want it to appear below your name

%----------------------------------------------------------------------------------------

\begin{document}

\maketitle

%----------------------------------------------------------------------------------------
%	TABLE OF CONTENTS
%----------------------------------------------------------------------------------------

%\setcounter{tocdepth}{1} % Uncomment this line if you don't want subsections listed in the ToC

\newpage

%----------------------------------------------------------------------------------------
%	PROBLEM 1
%----------------------------------------------------------------------------------------

% To have just one problem per page, simply put a \clearpage after each problem

\begin{homeworkProblem}

    In this problem, we implement a scheduler that schedules a set of processes, $P_1,P_2,\dots,P_n$, such that the processes start in cyclical order, $P_{i+1}$ starts after $P_i$ for each $1\le i<n$ and $P_{1}$ starts after $P_n$.
    
    The scheduler is made up of $n$ cells, where cell $i$ controls process $P_i$. A cell contains $4$ ports it uses for comunication. $in$ and $out$ ports are used to comunicate with other cells in the scheduler and $a$ and $b$ are used to signal the process to start and end execution. Each cell is defined using the following template:
    \[
        P = \overline{in}.a.out.b.P
    \]
    Using relabelling, we connect the cells together so that the $out$ port of cell $i$ is matched with the $in$ port of cell $i+1$ and $a$ and $b$ are uniquely defined for each process.
    \[
        P_i = P[c_{i-1}/in, a_i/a, c_i/out, b_i/b]
    \]
    where $c_0 = c_n$.

    Now the scheduler can simply run each cell in parallel. We restrict each $out$ and $in$ port since they must be matched by another cell but we allow $a$ and $b$ to communicate with the environment so we can monitor the actions of the scheduler. Each cell waits for a signal before it starts and so to prevent a deadlock, we send a single signal to the first cell which starts the scheduler. The scheduler can now be defined as follows:
    \[
        S = (P_1|P_2|\dots|P_n|c_1.nil)\backslash\{c_i|1\le i\le n\}
    \]
    


    There are two requirements that the scheduler must satisfy: Each process must start in a cyclical order ($P_{i+1}$ starts after $P_{i}$ and $P_1$ starts after $P_n$) and each cell must alternate between $a$ and $b$ actions. We will use weak bisimilarity to show that these properties hold.
    
    If we ignore all $b$ moves, then the scheduler should be weakly bisimilar to a process that simply performs actions $a_1,a_2,\dots,a_n$ repeadedly:
    \[
        cyclic = a_1.a_2.\dots.a_n.cyclic
    \]
    In order to ignore the $b$ moves of the shceduler, we must change it such that the $b$ moves cannot be observed. The way we do this is by defining an observer that runs in parallel with the scheduler and matches each $b$ move the scheduler performs.
    \[
        B = \sum_{i=1}^n\overline{b_i}.B
    \]
    We then restrict the $b$ moves so only the $a$ moves are visible but the execution of the scheduler is not affected. The definition for this new process is as follows:
    \[
        Y = (S|B)\backslash\{b_i|1\le i\le n\}
    \]
    Now $Y$ and $cyclic$ are weakly bisimilar if and only if $S$ performs the $a$ actions cyclically.

    For the second requirement, we will verify that a single cell must always alternate between $a$ and $b$. Instead of doing this for each cell in the scheduler, it is enough to show that it holds for the template they are instanciated from. Since we are only interesed in $a$ and $b$ actions, we must hide the $in$ and $out$ actions. Similar to before, we define a new process, $Z$, that can send and recieve $in$ and $out$ actions.
    \[
        Z = in.Z + \overline{out}.Z \\
    \]
    We then run $Z$ and the template, $P$, in parallel with the $in$ and $out$ actions restricted.
    \[
        W = (Z|P)\backslash\{in,out\}
    \]
    If the scheduler satisfies the property that each cell alternates between $a$ and $b$ actions, then $W$ should be weakly bisimilar to the following process:
    \[
        alternate = a.b.alternate
    \]
    Using the concurrency workbench, we can now easily verify that the properties hold by showing that $cyclic$ and $Y$ are weakly bisimilar and that $W$ and $alternate$ are weakly bisimilar. The CCS code for the processes is included in the file, task1.ccs, and so the properties can be verified using the following commands:
    \begin{align*}
        \texttt{cwb-nc>}&~\texttt{load task1.ccs}   \\
        \texttt{cwb-nc>}&~\texttt{eq Y Cycle}      \\
        \texttt{cwb-nc>}&~\texttt{eq W Alt}         \\
    \end{align*}

    Listing \ref{../task1.ccs} shows the CCS code for a scheduler with four processes.

    \perlscript{../task1.ccs}{Scheduler .CCS file}

    Another way to verify that the scheduler satisfies the two requirements is to express them as logical formulas and verify that the scheduler satisfies these formulas. In this report, we will use computational tree logic since the formulas are simpler and more readable, however these formulas will also be given as recursive Hennessy-Milner Logic formulas in the .mu files included with the report.
    
    We begin with the cyclical property. We can break it up into a number of smaller formulas, each of which examines a pair of cells. We must verify that every time the scheduler performs action $a_i$, then the next $a$ action it performs is $a_{i+1}$, for each $1\le i\le n$ where $a_{n+1} = a_1$. Another way to phrase this is that every time the scheduler makes an $a_i$ move, then it does not make an $a$ move until it makes an $a_{i+1}$ move. This property can be encoded as follows:
    \begin{align*}
        not\_a &= \bigwedge\limits_{i=1}^n\lnot EXa_i \\
        \phi_i &= AG([a_i](A(not\_a~U~EXa_{i+1}))) \\
    \end{align*}
    We must also make sure that the first $a$ action performed is $a_1$. This is encoded in the following formula
    \[
        \psi = A(not\_a~U~EXa_1)
    \]
    We can then combine these formulas into a single one that verifies that the scheduler is cyclical
    \[
        cyclic = \psi\land \bigwedge\limits_{i=1}^n \phi_i
    \]

    We now define a property that verifies that each cell in the scheduler alternates between $a$ and $b$ moves. Again we break this formula down into smaller units. We want to show that whenever a cell makes an $a$ move, it will not make an $a$ move again until it has made a $b$ move, and vice versa. We define the following formulas:
    \begin{align*}
        \phi_i = AG([a_i]A(\lnot EXa_i~U~EXb_i)) \\
        \psi_i = AG([b_i]A(\lnot EXb_i~U~EXa_i)) \\
    \end{align*}
    We then combine these formulas into a single one so that we verify this for each cell in the scheduler:
    \[
        alternate = \bigwedge\limits_{i=1}^n(\phi_i\land \psi_i)
    \]
    
    Both $cyclic$ and $alternate$ properties can be verified using the concurrency workbench. These properties are defined in the file, task1.mu, and we can use the following commands to verify them:
    \begin{align*}
        \texttt{cwb-nc>}&~\texttt{load task1.ccs}   \\
        \texttt{cwb-nc>}&~\texttt{load task1.mu}    \\
        \texttt{cwb-nc>}&~\texttt{chk S cyclical\_ctl}     \\
        \texttt{cwb-nc>}&~\texttt{chk S alternate\_ctl}  \\
    \end{align*}

    Listing \ref{../task1.mu} contains the formulas used to verify the properties of a scheduler of size 4.

    \perlscript{../task1.mu}{Scheduler .MU file}
    
    The .ccs and .mu files included with this report contain code for a scheduler of size $3,4$ and $5$ along with formulas used to verify that these schedulers perform as intended and using the concurrency workbench, we verified that they are correct. Each formula is also given in both CTL and recursive HML.

\end{homeworkProblem}

\newpage

%----------------------------------------------------------------------------------------
%	PROBLEM 2
%----------------------------------------------------------------------------------------

\begin{homeworkProblem}

  The second task at hand was to implement one-dimensional solitaire and to prove that it is solvable using the CCS concurrency workbench.
  The game was modelled as processes, where each tile is a process \(B(lack), W(hite),\) or \( E(mpty) \) and the board is a set of tiles (concurrent processes). 
  The three different colors were modelled as different processes. The processes were defined as templates and were created with appropriate relabellings to indicate location-based pairing between processes. 
	Solitaire has the helpful property that the $Empty$ tile must be swapped with whatever tile wants to swap, so we modelled the $Empty$ tile as having output ports for each direction and colored tiles to have input ports in their respective directions 
	($Black$ tiles can only go right and $White$ tiles can only go left) for the adjacent tile and its immediate neighbor.
	\emph{(Note: inputs and outputs can be freely interchanged here, since it is only the pairing that matters \-- $Empty$ tiles were selected as the output to improve readability in the source code.)}

	Listing \ref{../task2.ccs} demonstrates this functionality:

\perlscript{../task2.ccs}{1-Dimensional Solitaire .CCS file}
  Each template behaves differently based on what color it represents. Black tiles can jump either one or two tiles to the right depending on whether the $Empty$ tile is one or two tiles to the right (white tiles mirror this functionality in the opposite direction). Recall that $Empty$ tiles can swap with any adjacent tile.
  Relabelling the processes ensured that changing a process (for example, \(B \rightarrow W\)) allowed new input/output pairs to be matched.


  To transition from the configuration \([B_1|B_2|E_3|B_4|W_5|W_6|W_7]\) to \([B_1|B_2|W_3|B_4|E_5|W_6|W_7]\), $W_5$ must switch places with $E_3$ (by jumping over $B_4$).
  This move is made possible by the consistent location-dependent relabelling \-- recall that the board is modelled as concurrent processes.
  The relabelling on $E_3$ was \([b3/b, r4/rone, r5/rtwo, l3/l, r3/r, w3/w, l2/lone, l1/lone]\) when it was created as $B_3$, which means that $E_3$ exists as:
\[
	proc ~ E_3 = ~ \overline{r3}.B_3 + ~ \overline{l3}.W_3
\]
In a similar fashion, $W_5$ exists as:
\[
  proc ~ W_5 = ~ w_5.W_5 + ~ l4.E_5 + ~ l3.E_5
\]
Since these processes are running in parallel and restrictions are in place, $E_3$ is blocked until it is matched on either $'r3'$ or $'l3$. Incidentally, $W_5$ can match with $l3$, allowing the processes to swap 'colors' (the only other possibility would be for $W_5$ to indicate that tile 5 is white with the $w_5$ message).

In a similar fashion, if we had started in the configuration \([B_1|B_2|E_3|W_4|B_5|W_6|W_7]\), $W_4$ would be able to swap places with $E_3$ on $l3$ since $W_4$ would have been:
\[
	proc ~ W_4 = ~ w_4.W_4 + ~ l3.E_4 + ~ l2.E_4
\]

% CORRECTNESS AND VERIFICATION DETAILS BELOW

  Showing possible moves does not prove that the game is solvable (even though we can use these moves to solve it in our heads).
  To verify the correctness of the solitaire game, we simply asked if a configuration representing a solved game was reachable: 
  
\perlscript{../task2.mu}{1-Dimensional Solitaire .MU file}

  This elegant solution works due to the way that the game was modelled.
	Colored tiles (\(B_i, W_i\)) can always indicate what color occupies tile $i$ since $proc ~ Board$ does not restrict on $W_i$ (for any $i$).
	Therefore we can describe the property of a $solved$ game as ``Tiles $1-3$ are white and $5-7$ are black''.
	Using the fixed point-notation of Hennessy–Milner logic, we simply ask if the $solved$ state is reachable from the start configuration:
\shellcmd{
		chk Board is\_solvable
}

	\ldots which the concurrency workbench indicates as being $TRUE$.

% \problemAnswer{
% \begin{center}
% % \includegraphics[width=0.75\columnwidth]{example_figure} % Example image
% \end{center}
% 
% }
\end{homeworkProblem}

%----------------------------------------------------------------------------------------

\end{document}

