%%%%%%%%%%%%%%%%%%%%%%%%%%%%%%%%%%%%%%%%%
% Programming/Coding Assignment
% LaTeX Template
%
% This template has been downloaded from:
% http://www.latextemplates.com
%
% Original author:
% Ted Pavlic (http://www.tedpavlic.com)
%
% This template uses a Perl script as an example snippet of code, most other
% languages are also usable. Configure them in the "CODE INCLUSION 
% CONFIGURATION" section.
%
%%%%%%%%%%%%%%%%%%%%%%%%%%%%%%%%%%%%%%%%%

%----------------------------------------------------------------------------------------
%	PACKAGES AND OTHER DOCUMENT CONFIGURATIONS
%----------------------------------------------------------------------------------------

\documentclass{article}
%\documentclass[english,a4paper,twoside]{amsart}

\usepackage{fancyhdr} % Required for custom headers
\usepackage{lastpage} % Required to determine the last page for the footer
\usepackage{extramarks} % Required for headers and footers
\usepackage[usenames,dvipsnames]{color} % Required for custom colors
\usepackage{graphicx} % Required to insert images
\usepackage{listings} % Required for insertion of code
\usepackage{courier} % Required for the courier font

\usepackage[utf8]{inputenc}
\usepackage[T1]{fontenc}

% Margins
\topmargin=-0.45in
\evensidemargin=0in
\oddsidemargin=0in
\textwidth=6.5in
\textheight=9.0in
\headsep=0.25in

\linespread{1.1} % Line spacing

% Set up the header and footer
\pagestyle{fancy}
\lhead{\hmwkAuthorName} % Top left header
\chead{\hmwkClass\ (\hmwkClassInstructor): \hmwkTitle} % Top center head
\rhead{\firstxmark} % Top right header
\lfoot{\lastxmark} % Bottom left footer
\cfoot{} % Bottom center footer
\rfoot{Page\ \thepage\ of\ \protect\pageref{LastPage}} % Bottom right footer
\renewcommand\headrulewidth{0.4pt} % Size of the header rule
\renewcommand\footrulewidth{0.4pt} % Size of the footer rule

\setlength\parindent{0pt} % Removes all indentation from paragraphs

%----------------------------------------------------------------------------------------
%	CODE INCLUSION CONFIGURATION
%----------------------------------------------------------------------------------------

\definecolor{MyDarkGreen}{rgb}{0.0,0.4,0.0} % This is the color used for comments
\lstloadlanguages{Perl} % Load Perl syntax for listings, for a list of other languages supported see: ftp://ftp.tex.ac.uk/tex-archive/macros/latex/contrib/listings/listings.pdf
\lstset{%language=Perl, % Use Perl in this example
        frame=single, % Single frame around code
        basicstyle=\small\ttfamily, % Use small true type font
        keywordstyle=[1]\color{Blue}\bf, % Perl functions bold and blue
        keywordstyle=[2]\color{Purple}, % Perl function arguments purple
        keywordstyle=[3]\color{Blue}\underbar, % Custom functions underlined and blue
        identifierstyle=, % Nothing special about identifiers                                         
        commentstyle=\usefont{T1}{pcr}{m}{sl}\color{MyDarkGreen}\small, % Comments small dark green courier font
        stringstyle=\color{Purple}, % Strings are purple
        showstringspaces=false, % Don't put marks in string spaces
        tabsize=5, % 5 spaces per tab
        %
        % Put standard Perl functions not included in the default language here
        morekeywords={rand},
        %
        % Put Perl function parameters here
        morekeywords=[2]{on, off, interp},
        %
        % Put user defined functions here
        morekeywords=[3]{test},
       	%
        morecomment=[l][\color{Blue}]{...}, % Line continuation (...) like blue comment
        numbers=left, % Line numbers on left
        firstnumber=1, % Line numbers start with line 1
        numberstyle=\tiny\color{Blue}, % Line numbers are blue and small
        stepnumber=5 % Line numbers go in steps of 5
}

% Creates a new command to include a perl script, the first parameter is the filename of the script (without .pl), the second parameter is the caption
\newcommand{\perlscript}[2]{
\begin{itemize}
\item[]\lstinputlisting[caption=#2,label=#1]{#1}
\end{itemize}
}

%----------------------------------------------------------------------------------------
%	DOCUMENT STRUCTURE COMMANDS
%	Skip this unless you know what you're doing
%----------------------------------------------------------------------------------------

% Header and footer for when a page split occurs within a problem environment
\newcommand{\enterProblemHeader}[1]{
\nobreak\extramarks{#1}{#1 continued on next page\ldots}\nobreak
\nobreak\extramarks{#1 (continued)}{#1 continued on next page\ldots}\nobreak
}

% Header and footer for when a page split occurs between problem environments
\newcommand{\exitProblemHeader}[1]{
\nobreak\extramarks{#1 (continued)}{#1 continued on next page\ldots}\nobreak
\nobreak\extramarks{#1}{}\nobreak
}

\setcounter{secnumdepth}{0} % Removes default section numbers
\newcounter{homeworkProblemCounter} % Creates a counter to keep track of the number of problems

\newcommand{\homeworkProblemName}{}
\newenvironment{homeworkProblem}[1][Problem \arabic{homeworkProblemCounter}]{ % Makes a new environment called homeworkProblem which takes 1 argument (custom name) but the default is "Problem #"
\stepcounter{homeworkProblemCounter} % Increase counter for number of problems
\renewcommand{\homeworkProblemName}{#1} % Assign \homeworkProblemName the name of the problem
\section{\homeworkProblemName} % Make a section in the document with the custom problem count
\enterProblemHeader{\homeworkProblemName} % Header and footer within the environment
}{
\exitProblemHeader{\homeworkProblemName} % Header and footer after the environment
}

\newcommand{\problemAnswer}[1]{ % Defines the problem answer command with the content as the only argument
\noindent\framebox[\columnwidth][c]{\begin{minipage}{0.98\columnwidth}#1\end{minipage}} % Makes the box around the problem answer and puts the content inside
}

\newcommand{\homeworkSectionName}{}
\newenvironment{homeworkSection}[1]{ % New environment for sections within homework problems, takes 1 argument - the name of the section
\renewcommand{\homeworkSectionName}{#1} % Assign \homeworkSectionName to the name of the section from the environment argument
\subsection{\homeworkSectionName} % Make a subsection with the custom name of the subsection
\enterProblemHeader{\homeworkProblemName\ [\homeworkSectionName]} % Header and footer within the environment
}{
\enterProblemHeader{\homeworkProblemName} % Header and footer after the environment
}

%----------------------------------------------------------------------------------------
%	NAME AND CLASS SECTION
%----------------------------------------------------------------------------------------

\newcommand{\hmwkTitle}{Miniproject\ \#1} % Assignment title
\newcommand{\hmwkClass}{Modeling and Verification} % Course/class
\newcommand{\hmwkClassInstructor}{Anna Ingolfsdóttir} % Teacher/lecturer
\newcommand{\hmwkAuthorName}{Þröstur and Sævar} % Your name

%----------------------------------------------------------------------------------------
%	TITLE PAGE
%----------------------------------------------------------------------------------------

\title{
\vspace{2in}
\textmd{\textbf{\hmwkClass:\ \hmwkTitle}}\\
\vspace{0.1in}\large{\textit{\hmwkClassInstructor}}
\vspace{3in}
}

\author{\textbf{\hmwkAuthorName}}
\date{} % Insert date here if you want it to appear below your name

%----------------------------------------------------------------------------------------

\begin{document}

\maketitle

%----------------------------------------------------------------------------------------
%	TABLE OF CONTENTS
%----------------------------------------------------------------------------------------

%\setcounter{tocdepth}{1} % Uncomment this line if you don't want subsections listed in the ToC

\newpage

%----------------------------------------------------------------------------------------
%	PROBLEM 1
%----------------------------------------------------------------------------------------

% To have just one problem per page, simply put a \clearpage after each problem

\begin{homeworkProblem}

    The scheduler is made up of a $n$ cells, where cell $i$ controls $P_i$. A cell contains $4$ ports it uses for comunication. $in$ and $out$ ports are used to comunicate with other cells in the scheduler and $a$ and $b$ are used to signal the proces to start and end execution. Each cell is defined using the following template:
    \[
        P = \overline{in}.a.out.b.P
    \]
    We then connect the cells together using relabelling and the scheduler then simply runs each cell in parallel. We restrict the ports $in$ and $out$ since they must be matched by another cell but we allow $a$ and $b$ to communicate with the environment so we can monitor the actions the sche. Each cell initially waits for a signal before starting and so to prevent a deadlock, we send a single signal to the first cell which starts the scheduler.

    Listing \ref{../task1.ccs} shows the CCS code for a scheduler with four processes.

    \perlscript{../task1.ccs}{Scheduler .CCS file}

    There are two requirements that the scheduler must satisfy. The first requirement is that each process must start in a cyclical order, $P_{i+1}$ starts after $P_{i}$ and $P_1$ starts after $P_n$. The second requirement is that each cell must alternate between $a$ and $b$ actions.

    First we will show that if we ignore the $b$ moves, then the scheduler is bisimilar to a process that simply performs actions $a_1,a_2,\dots,a_n$ repeadedly:
    \[
        cyclic = a_1.a_2.\dots.a_n.cyclic
    \]
    The process can perform $a$ and $b$ moves and so we need to change it such that the $b$ moves cannot be observed. The way we do this is by defining an observer that runs in parallel with the scheduler and recieves the $b$ moves of the cells.
    \[
        B = \sum_{i=1}^n\overline{b_i}.B
    \]
    We then restrict the $b$ moves so only the $a$ moves are visible but the execution of the scheduler is not affected. The definition for this new process is as follows:
    \[
        Y = (S|B)\backslash\{b_i|1\le i\le n\}
    \]
    Using the cwb workbench, we can now easily verify that $cyclic$ and $Y$ are bisimilar.

    For the second requirement, we will verify that a single cell must always alternate between $a$ and $b$. Instead of doing this for each cell in the scheduler, it is enough to show that it holds for the template they are instanciated from. Since we are only interesed in $a$ and $b$ actions, we must hide the $in$ and $out$ actions. Similar as to before, we define a new process, $Z$, that can send and recieve $in$ and $out$ actions.
    \[
        Z = in.\overline{out}.Z \\
    \]
    And then run $Z$ and the template, $P$, in parallel with the $in$ and $out$ actions restricted.
    \[
        W = (Z|P)\backslash\{in,out\}
    \]
    If the scheduler satisfies the property that each cell alternates between $a$ and $b$ actions, then $W$ should be weakly bisimilar to the following process:
    \[
        alternate = \overline{a\vphantom{d}}.\overline{b}.alternate
    \]
    Again we use the cwb workbench to verify that $W$ and $alternate$ are weakly bisimilar.


    
    


    The scheduler was verifier with the file shown in listing \ref{../task1.mu} and foo bar bloo blah\ldots

\perlscript{../task1.mu}{Scheduler .MU file}

\end{homeworkProblem}

%----------------------------------------------------------------------------------------
%	PROBLEM 2
%----------------------------------------------------------------------------------------

\begin{homeworkProblem}

  The second task at hand was to implement one-dimensional solitaire and to prove that it is solvable using the CCS concurrency workbench.

\perlscript{../task2.ccs}{1-Dimensional Solitaire .CCS file}

  To verify the correctness of the solitaire game, we simply asked if a configuration representing a solved game was reachable: 
  
\perlscript{../task2.mu}{1-Dimensional Solitaire .MU file}

% \problemAnswer{
% \begin{center}
% % \includegraphics[width=0.75\columnwidth]{example_figure} % Example image
% \end{center}
% 
% }
\end{homeworkProblem}

%----------------------------------------------------------------------------------------

\end{document}

